\documentclass[12pt]{book}
\usepackage{amsmath, amssymb, amsthm}
\usepackage{latexsym, epsfig, ulem, cancel, multicol, hyperref}
\usepackage{graphicx, tikz, subfigure,pgfplots}
\usepackage[a4paper, total={6in, 8in}]{geometry}
\setlength{\parindent}{0pt}
\usepackage{multirow}
\usepackage{mathtools}
\pgfplotsset{width=10cm,compat=1.9}
\usepackage{esint}
\usepackage{authblk}

\title{A Calculus Note}
\author{Dennis Li \and Manu Dorghabekov}
\date{}



\newcommand{\liminfty}[1]{\lim_{#1 \to \infty}}
\newcommand{\limzero}[1]{\lim_{#1 \to 0}}
\newcommand{\Z}{\mathbb{Z}}
\newcommand{\R}{\mathbb{R}}
\newcommand{\C}{\mathbb{C}}
\newcommand{\lineint}[1]{\int_{#1}}
\newcommand{\pypx}[2]{\frac{\partial #1}{\partial #2}}
\newcommand{\divg}{\nabla \cdot}
\newcommand{\curl}{\nabla \times}
\newcommand{\dydx}[2]{\frac{d #1}{d #2}}
\newcommand{\sqbkt}[1]{\left[ #1 \right]}
\newcommand{\paren}[1]{\left( #1 \right)}
\newcommand{\tribkt}[1]{\left< #1 \right>}
\newcommand{\abso}[1]{\left|#1 \right|}


\begin{document}

\maketitle
\tableofcontents

\chapter{Calculus of Single Variable}

\section{Limits of Real Functions}

\subsection{Common Limits}

\[
\liminfty{x} \frac{x^{n-1} +c}{x^n + k} = 0 \;\;\;
\begin{cases}
    n \in + \mathbb{Z} \\
    c,k \in \mathbb{R}
\end{cases}
\]
\[
\liminfty{x} \frac{x^n +c}{x^{n+1}+k} = \infty \;\;\; 
\begin{cases}
    n \in + \mathbb{Z} \\
    c,k \in \mathbb{R}
\end{cases}
\]
\[
\limzero{x} \frac{\sin x}{x} = 1
\]
\[
\limzero{x} \frac{1 - \cos x}{x} =0
\]
\[
\limzero{x} \frac{e^x-1}{x} = 1
\]
\[
\liminfty{x} \left( 1 + \frac{1}{x}\right)^{\frac{1}{x}} = e
\]
\[
\limzero{x} (1+x)^{\frac{1}{x}} = e
\]
\[
\limzero{x} (1+\sin x)^{\frac{1}{x}} = e
\]

\section{Derivatives of Real Functions}
\subsection{Limit Definition of Derivatives}
\[
f'(x) = \limzero{h} \frac{f(x+h) - f(x)}{h}
\]
\subsection{Rules of Derivatives}
The linear nature of the derivative operator
\[
\dydx{}{x} \sqbkt{cf(x)} = cf'(x) \;\;\; c\in \R
\]
\[
\dydx{}{x} \sqbkt{f(x) \pm g(x)} = f'(x) \pm g(x) 
\]
Chain Rule
\[
\dydx{}{x}f\left( g(x) \right) = f'\left( g(x) \right)g'(x)
\]
Product Rule
\[
\dydx{}{x} \left[  f(x)g(x)  \right] = f'(x)g(x) + f(x)g'(x)
\]
Quotient Rule
\[
\dydx{}{x} \left[ \frac{f(x)}{g(x)} \right] = \frac{f'(x)g(x) - f(x)g'(x)}{g(x)^2}
\]
\subsection{Common Derivatives}
\subsubsection{Regular Derivatives}
\[
\dydx{}{x} c = 0 \;\;\; c\in\R
\]
\[
\dydx{}{x} x^n = nx^{n-1} \;\;\; n \in \R
\]
\[
\dydx{}{x} e^{ax} = ae^{ax}
\]
\[
\dydx{}{x} a^x = \ln{a}a^x
\]
\[
\dydx{}{x} \ln{|x|} = \frac{1}{x} \;\;\; x\neq 0
\]
\[
\dydx{}{x} \log_a x = \frac{1}{x\ln{a}} \;\;\; x>0
\]
\subsubsection{Hyperbolic Trigonometric Derivatives}
\[
\dydx{}{x} \sinh{x} = \cosh{x}
\]
\subsubsection{Trigonometric Derivatives}
\[
\dydx{}{x}\sin x = \cos x \;\;\; \dydx{}{x} \cos x = -\sin x
\]
\[
\dydx{}{x} \tan x = \sec ^2 x \;\;\; \dydx{}{x} \sec x = \sec x \tan x
\]
\[
\dydx{}{x} \cot x = - \csc ^2 x \;\;\; \dydx{}{x} \csc x = -\csc x \cot x
\]
\[
\dydx{}{x}  \arcsin x = \frac{1}{\sqrt{1-x^2}} \;\;\; \dydx{}{x} \arccos x =- \frac{1}{\sqrt{1-x^2}}
\]
\[
\dydx{}{x} \arctan x = \frac{1}{1+x^2} \;\;\; \dydx{}{x}\text{arccot } x = -\frac{1}{1+x^2}
\]
\[
\dydx{}{x} \sec ^{-1} x = \frac{1}{|x| \sqrt{x^2-1}} \;\;\; \dydx{}{x} \csc ^{-1} x =  - \frac{1}{|x| \sqrt{x^2-1}}
\]
\section{Integrals of a Single Variable}
\subsection{Fundamental Theorem of Calculus}

\[
\int_{a}^{b} f(x) dx = F(b) - F(a) \;\;\; \text{where} \;\;\; \dydx{}{x} F(x) = f(x)
\]
\[
\int f(x) dx = F(x) + c \;\;\; c \ \in \R
\]
\[
\int_{-\infty}^{\infty} f(x) dx = \liminfty{b} F(b) - \lim_{a \to -\infty} F(a)
\]
\subsection{Properties of Integrals}
\[
\int_{a}^{b} cf(x) dx = c\int_{a}^{b} f(x) dx \;\;\; c \in \R
\]
\[
\int_{a}^{b} f(x) \pm g(x) dx = \int_{a}^{b} f(x) dx \pm \int_{a}^{b} g(x) dx
\]
\[
\int_{a}^{b} f(x) dx = \int_{a}^{c} f(x) dx + \int_{c}^{b} f(x) dx
\]
\[
\int_{a}^{a} f(x) dx = 0
\]
\[
\int_{a}^{b} f(x) dx = - \int_{b}^{a} f(x) dx
\]
\subsection{Common Integrals}
\[
\int_{-\infty}^{\infty} e^{-x^2} dx= \sqrt{\pi}
\]
\[
\int \ln{x} \;dx = x\ln{x} - x +c
\]
\[
\int \frac{1}{a^2 + x^2} \; dx = \frac{1}{a} \tan^{-1} {\paren{\frac{x}{a}}} + c
\]
\[
\int \frac{1}{\sqrt{a^2 -x^2}} \; dx = \sin ^{-1} \paren{\frac{x}{a}} +c
\]
\subsection{Integration Methods}
\subsubsection{Integration by parts}
\[
\int udv = uv - \int vdu
\]
\subsubsection{Trig substitution}
\[
\sqrt{a^2 - b^2x^2} \Rightarrow x = \frac{a}{b}\sin{\theta} \Rightarrow \cos^2 \theta = 1- \sin ^2 \theta
\]
\[
\sqrt{b^2x^2 - a^2} \Rightarrow x = \frac{a}{b}\sec{\theta} \Rightarrow \tan^2 \theta =  \sec ^2 \theta -1
\]
\[
\sqrt{a^2 + b^2x^2} \Rightarrow x = \frac{a}{b}\tan{\theta} \Rightarrow \sec^2 \theta =  \tan ^2 \theta +1
\]
\subsubsection{Partial fraction decomposition}
Apply to
\[
\int \frac{P(x)}{Q(x)} \; dx
\]
\[
Q(x) = \paren{ax+b}^n \Rightarrow \frac{A_1}{ax+b} + \frac{A_2}{\paren{ax+b}^2} \ldots +\frac{A_n}{\paren{ax+b}^n}
\]
\[
Q(x) = \paren{ax^2+bx+c}^n \Rightarrow \frac{A_1x+B_1}{ax^2+bx+c} + \frac{A_2x+B_2}{\paren{ax^2+bx+c}^2} \ldots +\frac{A_nx+B_n}{\paren{ax^2+bx+c}^n}
\]
\subsubsection{I Integration Trick}
\[
I=\int_{-\infty}^{\infty} e^{-x^2} \; dx
\]
\[
I^2 = \int_{-\infty}^{\infty}\int_{-\infty}^{\infty} e^{-\paren{x^2+y^2}} \; dxdy
\]
\[
I^2 = \int_{0}^{2\pi}\int_{0}^{\infty} re^{-r^2} drd\theta = \pi
\]
\[
I = \sqrt{\pi}
\]
\section{Series and Taylor Series Expansion}
\subsection{Definition}
The Taylor series expansion of a function at point $a$ can be expressed as 
\[
f(x) = \sum_{n=0}^{\infty} \frac{f^{(n)}(a)}{n!}(x-a)^n
\]
\subsection{Taylor Series of Common Functions}
\[
e^x = \sum_{n=0}^{\infty} \frac{x^n}{n!} = 1+x+\frac{x^2}{2!}+\frac{x^3}{3!}+\ldots
\]
\[
\sin x = \sum_{n=0}^{\infty} \frac{(-1)^n x^{2n+1}}{(2n+1)!}=x-\frac{x^3}{3!}+\frac{x^5}{5!} - \ldots
\]
\[
\cos x = \sum_{n=0}^{\infty} \frac{(-1)^n x^{2n}}{(2n)!}=1-\frac{x^2}{2!}+\frac{x^4}{4!}- \ldots
\]
\[
\frac{1}{1-x} = \sum_{n=0}^{\infty} x^n \; \iff \; \{-1 < x < 1\}
\]
\[
(1+x)^p = \sum_{n=0}^{\infty} {p \choose x} x^n \;\;\; {p \choose n} = \frac{p!}{n!(p-n)!}
\]
\subsection{Common Infinite Sum}
\[
\sum_{k=m}^{n} z^k = \frac{z^m - z^{n+1}}{1-z}
\]
\[
\sum_{k=0}^{n} z^k = \frac{1 - z^{n+1}}{1-z}
\]
\[
\sum_{k=0}^{\infty} \frac{z^k}{k!} = e
\]
\[
\sum_{k=0}^{\infty} \frac{kz^k}{k!} = ze^z
\]
\[
\sum_{k=0}^{n} {n \choose k} = 2^n
\]
\[
\sum_{k=0}^{n} {n \choose k}^2 = {2n \choose n}
\]


\chapter{Multivariable Calculus}

\section{Multi-Dimensional Space}
\subsection{Notation}
In this section, $|\textbf{x}|$ would mean the magnitude of a vector, and \textbf{bold font} letter will denote a vector.\\
\newline
i.e. $\textbf{v}$ is a vector, and $|\textbf{v}|$ is its magnitude. 
\subsection{Vectors and coordinates}
\subsubsection{Definition}
A vector in space with dimension $n$ is defined as follows
\[
\forall \textbf{v} \in \R^n, \; n \in +\Z, \;\;\; \textbf{v}= \paren{
\begin{matrix}
    v_1\\
    v_2\\
    \vdots\\
    v_n
\end{matrix}
}
\]
Alternative notation for a vector
\[
\textbf{v} = \tribkt{v_1, v_2,\ldots,v_n}
\]
\subsubsection{Magnitude of a vector}
The magnitude of a vector is defined as follows
\[
\abso{\textbf{v}}=\textbf{v} \cdot \textbf{v} = \textbf{v}^T\textbf{v}=\sqrt{v_1^2 +v_2^2\ldots +v_n^2}
\]
\subsubsection{Unit vectors}
Unit vectors are vectors with magnitude of $1$, and are orthogonal to each other. For a space of $\R^3$, the unit vectors are
\[
\textbf{i} = \tribkt{1,0,0}
\]
\[
\textbf{j} = \tribkt{0,1,0}
\]
Some properties
\[
\textbf{i}\cdot \textbf{j} = 0
\]
Dot products between different unit vectors yields zero
\[
\textbf{i} \times \textbf{j} = \textbf{k} \;,\; \textbf{j} \times \textbf{i} = -\textbf{k}
\]
\[
\textbf{j} \times \textbf{k} = \textbf{i} \;,\; \textbf{k} \times \textbf{j} = -\textbf{i}
\]
\[
\textbf{k} \times \textbf{i} = \textbf{j} \;,\; \textbf{i} \times \textbf{k} = -\textbf{j}
\]
Cross products between unit vectors are as above. 

\subsubsection{Dot Product}
The dot product of a vector is defined as follows
\[
\forall \textbf{u},\textbf{v} \in \R^n, \;\;\; \textbf{u}\cdot \textbf{v} = u_1v_1 + u_2v_2 \ldots + u_nv_n
\]
notable properties
\[
\textbf{u}\cdot \textbf{v} = \textbf{v} \cdot \textbf{u}
\]
Dot products yields scalar. It means how much these 2 vectors works together. It can also be calculated as
\[
\textbf{u}\cdot \textbf{v} = |\textbf{u}||\textbf{v}|\cos{\theta}
\]
Where $\theta$ is the angle between the 2 vector.\\
\newline
Let the dot product of 2 vectors be denoted by $k$, it has several information about the original vectors
\[
\begin{cases}
    k<0 \rightarrow \text{opposite direction}\\
    k=0 \rightarrow \text{orthogonal}\\
    k>0 \rightarrow \text{same direction}\\
\end{cases}
\]

\subsubsection{Cross Product}
For our purpose of calculus, we only define cross product for vectors in $\R^3$
\[
\textbf{u} = \tribkt{x,y,z} \;\;\; \textbf{v} = \tribkt{P,Q,R}
\]
And then the cross product is defined as a determinant of this matrix
\[
\textbf{u} \times \textbf{v} = 
\abso{
\begin{matrix}
    i&j&k\\
    x&y&z\\
    P&Q&R
\end{matrix}
}= \paren{Ry-Qz}\textbf{i} - \paren{Rx-Pz}\textbf{j} + \paren{Qx-Py}\textbf{k}
\]
or with vector notation
\[
\textbf{u}\times\textbf{v}=\tribkt{Ry-Qz,\; Rx-Pz,\; Qx-Py}
\]
\subsubsection{Normalizing a vector}
Normalizing a vector means to make it such that its dot product is $1$. And to obtain that we simply do
\[
\hat{\textbf{v}} = \frac{\textbf{v}}{\abso{\textbf{v}}}
\]
$\hat{\textbf{v}}$ means the normalized vector or $\textbf{v}$.
\subsubsection{Projection}
A vector can be projected onto another vector with the following operation
\[
\textbf{P}_\textbf{u} = \frac{ \textbf{v}^T \textbf{u} }{ \textbf{v}^T\textbf{v}}\textbf{v} = \frac{\textbf{v}\cdot \textbf{u}}{\abso{\textbf{v}}}\textbf{v}
\]
$\textbf{P}_\textbf{u}$ is the projection of $\textbf{u}$ on $\textbf{v}$

\section{Vector Functions}
\subsection{Vector Function}
A curve in 3-space can be expressed as a vector whose components are a function of a single variable. The curve is the trace that the vector would leave in the domain for $t$.
\[
\textbf{r}(t) = \tribkt{f(t),g(t),h(t)}
\]
If the \textit{curve} is a straight line, it can also be expressed as
\[
\textbf{r}(t) = \tribkt{x_0,y_0,z_0} + t\tribkt{a,b,c} 
\]
Where $a,b,c$ are constants, representing the slope of the line, and $x_0,y_0,z_0$ is the point which the line crosses. \\
\newline
Curves can also be expressed in a way as demonstrated below
\[
x^2+y^2=1
\]
This is a circle in on the xy-plane $(z=0)$. And it can be parameterized to a single variable vector function
\[
\textbf{r}(t) = \tribkt{\cos t, \sin t, 0}
\]
Taking derivatives of a vector function is straight forward.
\[
\text{let }\textbf{r}(t) = \tribkt{f(t),g(t),h(t)}
\]
\[
\text{then } \textbf{r}'(t) = \tribkt{f'(t),g'(t),h'(t)}
\]
\[
\text{and } \int \textbf{r}'(t) dt = \tribkt{f(t),g(t),h(t)} + \textbf{C}
\]
Where $\textbf{C}$ is a vector of 3 components that contains all 3 integration constants. 

\subsection{Arc length of curve}
The arc length of a generic vector function $\textbf{r}(t) = \tribkt{f(t),g(t),h(t)}$ can be calculated as 
\[
\int_{a}^{b} \abso{\textbf{r}'(t)} dt
\]
And a function that gives you the arc length would simply be 
\[
s(x) = \int_{a}^{x} \abso{\textbf{r}'(t)} dt
\]
\subsection{Curvature of a curve}
To obtain the curvature, we have to go through the following process.\\
\newline
First, find
\[
s(x) = \int_{a}^{x} \abso{\textbf{r}'(t)} dt
\]
Secondly, find $x(s)$, or the inverse of the arc length function.\\
\newline
Thirdly, we find the tangent vector
\[
\textbf{T} = \frac{\textbf{r}'(t)}{\abso{\textbf{r}'(t)}} 
\]
Fourth step, we find \textbf{T} in terms of \textbf{s}, and lastly the curvature can be given by
\[
\kappa = \abso{\frac{d\textbf{T}(s)}{ds}}=\frac{\|\mathbf{r}'(t) \times \mathbf{r}''(t)\|}{\|\mathbf{r}'(t)\|^3}
\]
\subsection{Find Osculating Plane}
A Osculating Plane can be found with a bit more steps. We starts with 
\[
\textbf{T} = \frac{\textbf{r}'(t)}{\abso{\textbf{r}'(t)}} 
\]
Then, we find the unit normal vector to the tangent vector
\[
\textbf{N} = \frac{\textbf{T}'}{\abso{\textbf{T}'}}
\]
We then find the binormal vector, which will be the normal vector of the osculating plane. 
\[
\textbf{B} = \textbf{T} \times \textbf{N}
\]
\section{Multivariable Function}
A multivariable function is a function with multiple variable. Examples being
\[
f(x,y,z) = x+y+z
\]
\[
g(x,y) = cosx+siny
\]

\subsection{Notations}
\[
\pypx{}{x}f(x,y) = f_x(x,y) \;\;\; \pypx{}{y}f(x,y) = f_y(x,y) 
\]
\[
\pypx{^2}{x^2}f(x,y) = f_{xx}(x,y)  \;\;\; \pypx{^2}{y^2}f(x,y) = f_{yy}(x,y) 
\]
\[
\pypx{}{x \partial y} = f_{xy}(x,y)  \;\;\; \pypx{}{y \partial x} = f_{yx}(x,y) 
\]
\[
\textbf{F}(x,y,z) = \Vec{F}(x,y,z)
\]
\subsection{Basic Partial Derivative Properties}
\[
\dydx{}{x} f(x,y) = Undefined
\]
\[
\pypx{}{x} f(x,y) = f_x(x) \;\;\; \pypx{^2}{x^2} f(x,y) = f_{xx} (x,y)
\]
\[
\pypx{}{y} f(x,y) = f_y(y) \;\;\; \pypx{^2}{y^2} f(x,y) = f_{yy} (x,y)
\]
\subsubsection{Commutativity of partial derivative}
If $f(x,y)$ is defined in a disk $D$, and $f_xy,f_yx$ are continuous in $D$, then
\[
f_xy = f_yx
\]
\subsection{Directional Derivative}
Let $\textbf{u} = \tribkt{a,b}$ be a unit vector, then the directional derivative is of a function $f(x,y)$ can be expressed as 
\[
D_\textbf{u}f(x,y) = \tribkt{f_x,f_y}\cdot \tribkt{a,b}
\]

\subsection{Chain Rule in Partial Derivatives}
\[
z = f(x,y) \;\;\ \; x = g(t) \;\;\; y = h(t)
\]
\[
\dydx{z}{t} = \pypx{f}{x}\dydx{x}{t} + \pypx{f}{y}\dydx{y}{t}
\]
\subsection{Gradient}
A gradient of a multivariable function is a vector field that describes how this function is changing. It gives you the rate of change on all direction throughout the function. 
\[
\nabla f(x,y,z) = \textbf{F}(x,y,z)=\tribkt{ \pypx{f}{x}, \pypx{f}{y}, \pypx{f}{z}}
\]
A special property is that, the vector field that is obtained this way will produce vectors that is perpendicular to the level curve of the function. 

\subsection{Jacobian}
Suppose a function $f(x,y,z) = P+Q+R$, where $P,Q,R$ are functions of $x,y,z$, then we can obtain its Jacobian matrix by
\[
\begin{bmatrix}
    P_x & P_y & P_z\\
    Q_x & Q_y & Q_z\\
    R_x & R_y & R_z
\end{bmatrix}
\]
Taking the determinant, we can obtain the \textbf{Jacobian} by taking the determinant of the matrix
\[
\mathbf{J} = \abso{
\begin{matrix}
    P_x & P_y & P_z\\
    Q_x & Q_y & Q_z\\
    R_x & R_y & R_z
\end{matrix}
}
\]


\subsection{Optimization}
If we want to find the max and min of a multivariable function


\section{Vector Calculus}
\subsection{Vector Field}
We say a vector field is conservative if
\[
\textbf{F}(x,y,z) = \nabla f(x,y,z)
\]
we call $f(x,y,z)$ the \textbf{potential} of $\textbf{F}$

\subsection{Divergence and Curl}
\[
\textbf{F}(x,y,z) = \tribkt{P,Q,R} \;\;\;
\]
$P,Q,R$ are functions of $(x,y,z)$
\[
\divg \textbf{F} = \pypx{P}{x} + \pypx{Q}{y} + \pypx{R}{z}
\]
\[
\curl \textbf{F} = \tribkt{R_y - Q_z,P_z - R_x,Q_x - P_y}
\]
\subsection{Product Rules}
There are six product rules, two for gradients:

(i) \[
\nabla (fg) = f \nabla g + g \nabla f,
\]

(ii) \[
\nabla (\mathbf{A} \cdot \mathbf{B}) = \mathbf{A} \times (\nabla \times \mathbf{B}) + \mathbf{B} \times (\nabla \times \mathbf{A}) + (\mathbf{A} \cdot \nabla)\mathbf{B} + (\mathbf{B} \cdot \nabla)\mathbf{A},
\]

two for divergences:

(iii) \[
\nabla \cdot (f \mathbf{A}) = f (\nabla \cdot \mathbf{A}) + \mathbf{A} \cdot (\nabla f),
\]

(iv) \[
\nabla \cdot (\mathbf{A} \times \mathbf{B}) = \mathbf{B} \cdot (\nabla \times \mathbf{A}) - \mathbf{A} \cdot (\nabla \times \mathbf{B}),
\]

and two for curls:

(v) \[
\nabla \times (f \mathbf{A}) = f (\nabla \times \mathbf{A}) - \mathbf{A} \times (\nabla f),
\]

(vi) \[
\nabla \times (\mathbf{A} \times \mathbf{B}) = (\mathbf{B} \cdot \nabla)\mathbf{A} - (\mathbf{A} \cdot \nabla)\mathbf{B} + \mathbf{A} (\nabla \cdot \mathbf{B}) - \mathbf{B} (\nabla \cdot \mathbf{A}).
\]


\subsection{Fundamental Theorem for Gradients}
\[
\int_{a}^{b} (\nabla T) \cdot d\mathbf{l} = T(b) - T(a)
\]
Where $T(x, y, z)$ is a scalar function, points $\mathbf{a}$ and $\textbf{b}$ are the beginning and end points of the path taken. Note, the path independence $\oint (\nabla T) \cdot d\mathbf{l} = 0
$. 

\subsection{Fundamental Theorem for Divergences}
\[
\int_V (\nabla \cdot \mathbf{v}) \, d\tau = \oint_S \mathbf{v} \cdot d\mathbf{a}
\]
Notice that the boundary term is itself an integral (specifically, a surface
integral). This is reasonable: the “boundary” of a line is just two end points, but
the boundary of a volume is a (closed) surface.

\subsection{Fundamental Theorem for Curls}
\[
\int_S (\nabla \times \mathbf{v}) \cdot d\mathbf{a} = \oint_P \mathbf{v} \cdot d\mathbf{l}
\]
Note that $\int_S (\nabla \times \mathbf{v}) \cdot d\mathbf{a}$ is independent of the particular surface, only depends on the boundary line. $\oint_S (\nabla \times \mathbf{v}) \cdot d\mathbf{a}=0$ for any closed surface, since the
boundary line, like the mouth of a balloon, shrinks down to a point, and hence the right side of equation goes to zero.
\subsection{Curvilinear Coordinates}
\subsubsection{Spherical Coordinates}
Their relation to Cartesian coordinates:
\[
x = r \sin \theta \cos \varphi, \quad
y = r \sin \theta \sin \varphi, \quad
z = r \cos \theta.
\]
The unit vectors $\mathbf{\hat{r},\hat{\theta}, \hat{\varphi}}$ in terms or Cartesian unit vectors:
\begin{align*}
\hat{r} &= \sin \theta \cos \varphi \, \hat{x} + \sin \theta \sin \varphi \, \hat{y} + \cos \theta \, \hat{z}, \\
\hat{\theta} &= \cos \theta \cos \varphi \, \hat{x} + \cos \theta \sin \varphi \, \hat{y} - \sin \theta \, \hat{z}, \\
\hat{\varphi} &= -\sin \varphi \, \hat{x} + \cos \varphi \, \hat{y}.
\end{align*}
Beware of differentiating a vector that is expressed in
spherical coordinates, since the unit vectors themselves are functions of position. Thus the general infinitesimal displacement \( d\mathbf{l} \) is

\[
d\mathbf{l} = dr \, \mathbf{\hat{r}} + r \, d\theta \, \mathbf{\hat{\theta}} + r \sin \theta \, d\varphi \, \mathbf{\hat{\varphi}}.
\]

Here are the vector derivatives in spherical coordinates:

\textbf{Gradient:}

\[
\nabla T = \frac{\partial T}{\partial r} \mathbf{\hat{r}} + \frac{1}{r} \frac{\partial T}{\partial \theta} \mathbf{\hat{\theta}} + \frac{1}{r \sin \theta} \frac{\partial T}{\partial \varphi} \mathbf{\hat{\varphi}}
\]

\textbf{Divergence:}

\[
\nabla \cdot \mathbf{v} = \frac{1}{r^2} \frac{\partial}{\partial r} (r^2 v_r) + \frac{1}{r \sin \theta} \frac{\partial}{\partial \theta} (\sin \theta v_\theta) + \frac{1}{r \sin \theta} \frac{\partial v_\varphi}{\partial \varphi}
\]

\textbf{Curl:}

\[
\nabla \times \mathbf{v} = \frac{1}{r \sin \theta} \left( \frac{\partial}{\partial \theta} (\sin \theta v_\varphi) - \frac{\partial v_\theta}{\partial \varphi} \right) \mathbf{\hat{r}} + \frac{1}{r} \left( \frac{1}{\sin \theta} \frac{\partial v_r}{\partial \varphi} - \frac{\partial}{\partial r} (r v_\varphi) \right) \mathbf{\hat{\theta}} + \frac{1}{r} \left( \frac{\partial}{\partial r} (r v_\theta) - \frac{\partial v_r}{\partial \theta} \right) \mathbf{\hat{\varphi}}
\]

\textbf{Laplacian:}

\[
\nabla^2 T = \frac{1}{r^2} \frac{\partial}{\partial r} \left( r^2 \frac{\partial T}{\partial r} \right) + \frac{1}{r^2 \sin \theta} \frac{\partial}{\partial \theta} \left( \sin \theta \frac{\partial T}{\partial \theta} \right) + \frac{1}{r^2 \sin^2 \theta} \frac{\partial^2 T}{\partial \varphi^2}
\]

\subsubsection{Cylindrical Coordinates}
Their relation to Cartesian coordinates:
\[
x = s \cos \varphi, \quad
y = s \sin \varphi, \quad
z = z.
\]
The unit vectors $\mathbf{\hat{s}, \hat{\varphi}, \hat{z}}$ in terms or Cartesian unit vectors:
\begin{align*}
\mathbf{\hat{s}} &= \cos \varphi \, \mathbf{\hat{x}} + \sin \varphi \, \mathbf{\hat{y}}, \\
\mathbf{\hat{\varphi}} &= -\sin \varphi \, \mathbf{\hat{x}} + \cos \varphi \, \mathbf{\hat{y}}, \\
\mathbf{\hat{z}} &= \mathbf{\hat{z}}.
\end{align*}
The infinitesimal displacement:
\[
d\mathbf{l} = ds \, \mathbf{\hat{s}} + s \, d\varphi \, \mathbf{\hat{\varphi}} + dz \, \mathbf{\hat{z}}
\]

The vector derivatives in cylindrical coordinates are:

\textbf{Gradient:}

\[
\nabla T = \frac{\partial T}{\partial s} \, \mathbf{\hat{s}} + \frac{1}{s} \frac{\partial T}{\partial \varphi} \, \mathbf{\hat{\varphi}} + \frac{\partial T}{\partial z} \, \mathbf{\hat{z}}
\]

\textbf{Divergence:}

\[
\nabla \cdot \mathbf{v} = \frac{1}{s} \frac{\partial}{\partial s} (s v_s) + \frac{1}{s} \frac{\partial v_{\varphi}}{\partial \varphi} + \frac{\partial v_{z}}{\partial z}
\]

\textbf{Curl:}

\[
\nabla \times \mathbf{v} = \left( \frac{1}{s} \frac{\partial v_{z}}{\partial \varphi} - \frac{\partial v_{\varphi}}{\partial z} \right) \mathbf{\hat{s}} + \left( \frac{\partial v_{s}}{\partial z} - \frac{\partial v_{z}}{\partial s} \right) \mathbf{\hat{\varphi}} + \frac{1}{s} \left( \frac{\partial}{\partial s} (s v_{\varphi}) - \frac{\partial v_{s}}{\partial \varphi} \right) \mathbf{\hat{z}}
\]

\textbf{Laplacian:}

\[
\nabla^2 T = \frac{1}{s} \frac{\partial}{\partial s} \left( s \frac{\partial T}{\partial s} \right) + \frac{1}{s^2} \frac{\partial^2 T}{\partial \varphi^2} + \frac{\partial^2 T}{\partial z^2}
\]


\end{document}

